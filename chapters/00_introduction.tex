\cleardoublepage%
\chapter{\label{chap:intro}Introduction}%

\noindent The Standard Model of elementary particles (SM) is a quantum field theory, developed in the 1970s, describing the elementary particles and their interactions. Despite its successes in explaining various phenomena, the SM leaves several key questions unanswered, including the nature of dark matter, neutrino oscillations \cite{Fukuda_1998,Ahmad_2002,Agafonova_2018}, the hierarchy problem, and the strong CP problem. These limitations underscore the need for models beyond the Standard Model (BSM) to comprehensively account for observed phenomena in particle physics.\\
\indent Experimental validation of BSM theories often relies on probing the properties of SM particles for deviations or anomalies. Such investigations are conducted at high-energy colliders like the Large Hadron Collider (LHC) at CERN, which provides a fertile ground for exploring new physics. With the discovery of the Higgs boson in 2012 by the ATLAS and the CMS collaborations with a mass of approximately 125 GeV \cite{Higgs1, Higgs2, Higgs3}, the last component of the SM was found. A chain of experiments have been undertaken to study its properties at the LHC.\\
\indent This project takes part in the CMS collaboration, aiming to probe the production of the Standard Model Higgs boson associated with a top quark pair (t$\overline{\text{t}}$H). Specifically, t$\overline{\text{t}}$H has been established as a powerful channel in the search of H $\rightarrow$ b$\overline{\text{b}}$. As technology advances, our attention turns to even more challenging searches, such as the H $\rightarrow$ c$\overline{\text{c}}$ decay, which represents a crucial test of the fermion mass generation mechanism in the Standard Model.\\
\indent In this context, the associated production of top and bottom quark-antiquark pairs, t$\overline{\text{t}}+$b$\overline{\text{b}}/$c$\overline{\text{c}}$, emerges as a leading background. The understanding of the t$\overline{\text{t}}+$b$\overline{\text{b}}/$c$\overline{\text{c}}$ process is important in this measurement, as an accurate and reliable description of the t$\overline{\text{t}}+$b$\overline{\text{b}}/$c$\overline{\text{c}}$ process will allow for a measurement of the signal process under scrutiny with high accuracy. \\ 
\indent This thesis is structured as follows: Chapter \ref{chap:first} presents an introduction of the theoretical background. In chapter \ref{chap:Experiment}, the LHC and CMS experiment are introduced. Chapter \ref{chap:gen} gives a short overview of the event simulation procedure.\\
\indent In chapter \ref{chap:res}, a MC generator study for t$\overline{\text{t}}+$b$\overline{\text{b}}$ background process modelling is presented. Some MC generator free parameters are varied and the sensitivity of the events on these free parameters are analysed. The aim of this study is to obtain a better understanding of this important and irreducible background process for the t$\overline{\text{t}}$H(b$\overline{\text{b}}$/c$\overline{\text{c}}$) search and thereby improve the sensitivity of measuring this process.