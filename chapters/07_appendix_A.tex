\cleardoublepage%
\refstepcounter{chapter}%
% \phantomsection
\chapter[Appendix A - \textsc{Powheg Box Res} t$\overline{\text{t}}$+b$\overline{\text{b}}$ Configuration]{\label{chap:apx_exp_data}Appendix A \\
{\fontfamily{bch}\fontshape{sc}\selectfont{Powheg Box Res}} Configuration}
\chaptermark{Appendix A}%
\section{Collider setup}
\begin{lstlisting}[basicstyle=\ttfamily\scriptsize,frame = single]
ebeam1 6500d0     ! energy of beam 1
ebeam2 6500d0     ! energy of beam 2
ih1   1           ! hadron 1 (1 for protons, -1 for antiprotons)
ih2   1           ! hadron 2 (1 for protons, -1 for antiprotons)
\end{lstlisting}
\section{Parameters for the generation of spin correlated t$\overline{\text{t}}$ decays}
\begin{lstlisting}[basicstyle=\ttfamily\scriptsize,frame = single]
#tdec/wmass 80.4      ! W mass for top decay
#tdec/wwidth 2.141
#tdec/bmass 4.8
#tdec/twidth  1.31    ! 1.33 using PDG LO formula
#tdec/elbranching 0.108
#tdec/emass 0.00051
#tdec/mumass 0.1057
#tdec/taumass 1.777
#tdec/dmass   0.100
#tdec/umass   0.100
#tdec/smass   0.200
#tdec/cmass   1.5
#tdec/sin2cabibbo 0.051
\end{lstlisting}
\newpage
\section{Integrator and event generator settings }
\begin{lstlisting}[basicstyle=\ttfamily\scriptsize,frame = single]
! values below suitable for a manyseeds run with 128 cores and 3 xgriditerations
ncall1  40000    ! number of calls for initializing the integration grid
itmx1    2       ! number of iterations for initializing the integration grid
ncall2  40000    ! number of calls for computing the integral and finding upper bound
ncall2rm  80000  ! number of calls for computing the integral and finding upper bound
itmx2    3       ! number of iterations for computing the integral and finding upper
                 ! bound

foldcsi   5      ! number of folds on csi integration
foldy     5      ! number of folds on  y  integration
foldphi   5      ! number of folds on phi integration

nubound 10000    ! number of bbarra calls to setup norm of upper bounding function

icsimax  1       ! <= 100, number of csi subdivision when computing the upper bounds
iymax    1       ! <= 100, number of y subdivision when computing the upper bounds
xupbound 2d0     ! increase upper bound for radiation generation
storemintupb 1   ! (powheg default 0, ttbb recommended 1) 
                 ! 1 ... store st2 btilde calls to set up upper bounding envelope;
                 ! 0 ... do not 
fastbtlbound 1   ! (powheg default 0, ttbb recommended 1) 
                 ! 1 ... fast calculation of the btilde upper bounding envelope;
                 ! 0 ... usual calculation of the btilde upper bounding
compress_upb 1   ! uses zlib to compress the upper bounding envelope on the fly
compress_lhe 1   ! uses zlib to compress the .lhe files, compressed .lhe files can be 
                 ! inspected using zcat
use-old-grid 1   ! (powheg default 0) 
                 ! 1 ... use old grid if file pwggrids.dat is present;
                 ! 0 ... regenerate, must be 1 for a manyseed run
use-old-ubound 1 ! (powheg default 0)  1 ... use norm of upper bounding function 
                 ! stored in pwgubound.dat, if present;
                 ! 0 ... regenerate, must be 1 for a manyseed run
\end{lstlisting}
\section{Scale settings}
\begin{lstlisting}[basicstyle=\ttfamily\scriptsize,frame = single]
runningscales 2 ! (powheg default 0, ttbb recommended 2) renormalization and 
                ! factorization scale setting
                ! 0... mur=muf=2*mtop; 
                ! 1... mur=[mT(top)*mT(tbar)*mT(b)*mT(bbar)]**(1/4), 
                !      muf=1/2*[mT(top)+mT(tbar)+mT(b)+mT(bbar)+mT(gluon)];
                ! 2... mur=1/2*[mT(top)*mT(tbar)*mT(b)*mT(bbar)]**(1/4), 
                !      muf=1/4*[mT(top)+mT(tbar)+mT(b)+mT(bbar)+mT(gluon)];
btlscalereal 1  ! (powheg default 0, ttbb recommended 2) let's user chose what  
                ! kinematics (real or uborn) to use for the scales in real matrix 
                ! element, by default uborn is used
btlscalect 1    ! (powheg default 0, ttbb recommended 2) let's user chose what
                ! kinematics (real or uborn) to use for the scales in the real
                ! counterterm, by default same as real are used

\end{lstlisting}
\section{Damping}
\begin{lstlisting}[basicstyle=\ttfamily\scriptsize,frame = single]
withdamp 1         ! (powheg default 0, ttbb recommended 1) activate separation of
                   ! the real cross section into the singular and remnant
                   ! contributions through a damping with h^2/(pt2+h^2)
#hdamp 172.5       ! fixed value of h, in the expression in the withdamp entry,
                   ! not recommended for ttbb
#dynhdamp 1        ! (ttbb default 1) 1 ... calculate the on per event  basis as
                   !  h = sqrt(1/2)*(E[t]+E[t~])*dynhdampPF with
                   ! E[x]=sqrt(m[x]**2+pt[x]**2), 
                   ! 0 ... use static h with the value set in hdamp
#dynhdampPF 0.5    ! (ttbb default 0.5) the value of dynhdampPF from above
#hdampMassTh 5.0   ! (ttbb default 5) by default in powheg, massive emitters are not
                   ! subjected to damping, this introduces damping for massive emitters
                   ! lighter than hdampMassTh (focusing in particular on bottom quarks)
#bornzerodampcut 2 ! (powheg defaul 5, ttbb default 2) points in which the real matrix
                   ! element is larger than bornzerodampcut x
                   ! [its collinear approximation] are considered remnant
\end{lstlisting}
\section{Scale and PDF reweighting}
\begin{lstlisting}[basicstyle=\ttfamily\scriptsize,frame = single]
rwl_file 'pwg-rwl.dat'  ! If set to '-' read the xml reweighting info 
                        ! from this same file. Otherwise, it specifies 
                        ! the xml file with weight information
#<initrwgt>
#<weight id='1'>default</weight> ! Default weight, necessary when using
                                 ! for reweighting 1, in order to save the
                                 ! weight including the virtual corrections
#<weight id='2' > renscfact=2d0 facscfact=2d0 </weight>
#<weight id='3' > renscfact=0.5d0 facscfact=0.5d0 </weight>
#<weight id='4' > renscfact=1d0 facscfact=2d0 </weight>    
#<weight id='5' > renscfact=1d0 facscfact=0.5d0 </weight>  
#<weight id='6' > renscfact=2d0 facscfact=1d0 </weight>    
#<weight id='7' > renscfact=0.5d0 facscfact=1d0 </weight>  
#</initrwgt>
rwl_group_events 1000 ! (powheg default 1000) it keeps 1000 events in 
                      ! memory, reprocessing them together for reweighting

manyseeds 1
\end{lstlisting}
\newpage
\section{Parallel runs settings}
\begin{lstlisting}[basicstyle=\ttfamily\scriptsize,frame = single]
manyseeds 1       ! (default 0) 1 ... perform a manyseeds run
#xgriditeration 1 ! this controls xgriditeration in stage 1 manyseeds runs, modify
                  ! correspondingly throughout the run either here or via a command
                  ! line argument as ./pwhg_main xgriditeration=1 
#parallelstage  1 ! this controls stage in manyseeds run, modify either here
                  ! correspondingly throughout the run or via a command line
                  ! argument as ./pwhg_main parallelstage=1 
maxseeds 1000     ! maximum number of cores to consider

\end{lstlisting}
\section{Other settings}
\begin{lstlisting}[basicstyle=\ttfamily\scriptsize,frame = single]
clobberlhe 1         ! 1 ... delete the event file if it exists, 
                     ! 0 ... exit if it exists
olverbose 1          ! set the OpenLoops verbosity level (see OL manual)
alphas_from_lhapdf 1 ! (powheg default 0, ttbb recommended 1) use the LHAPDF routine
                     ! for alphaS running instead of the powheg internal routine
                     ! (everywhere, except for in the Sudakov form factor)

for_reweighting 0    ! (powheg default 0, ttbb recommended 0) 
                     ! 1... calculate with virtual corrections switched off throughout
                     ! and then reweight with virtual corrections switched on
                     ! (useful for when virtual corrections are costly to calculate).
                     ! Note that the events will have two weights, the first an
                     ! intermediate weight (not to be used), and the second one
                     ! including all the contributions

!!!!!!!!!!!!!!!!!!!!!!!!!!!!!!!!!!!!!!!!!!!!!!!!!!!!!!!!!!!!!!!!!!!!!!!!!!!!!!!!!!!!!!
! Physics constants 
!!!!!!!!!!!!!!!!!!!!!!!!!!!!!!!!!!!!!!!!!!!!!!!!!!!!!!!!!!!!!!!!!!!!!!!!!!!!!!!!!!!!!!
#tmass 172.5d0   ! (ttbb default 172.5)
#bmass 4.75d0    ! (ttbb default 4.75) bottom quark mass, setting it to zero
                 ! introduces a singularity in the born matrix element, make
                 ! sure to adjust hdampMassTh correspondingly if using value
                 ! larger than 5.0

\end{lstlisting}



