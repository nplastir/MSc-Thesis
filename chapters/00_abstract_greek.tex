\newpage%
\clearpage%
\selectlanguage{greek}
\chapter*{{\fontfamily{cmr}\fontshape{sc}\selectfont Περίληψη}}
\markboth{\MakeUppercase{{\fontfamily{cmr}\selectfontΠερίληψη}}}{\MakeUppercase{{\fontfamily{cmr}\selectfontΠερίληψη}}}%
\addcontentsline{toc}{chapter}{{\fontfamily{cmr}\selectfontΠερίληψη}}%
\phantomsection
{\fontfamily{cmr}\fontseries{m}\selectfont{
\noindent Η μελέτη της συσχετισμένης παραγωγής ζευγών \foreignlanguage{english}{top} και \foreignlanguage{english}{bottom} κουάρκ-αντικουάρκ \foreignlanguage{english}{(t$\overline{\text{t}}+$b$\overline{\text{b}}$)}, σε συγκρούσεις πρωτονίου-πρωτονίου \foreignlanguage{english}{(p-p)} στον \foreignlanguage{english}{LHC} αποτελεί σημαντική πρόκληση λόγω της μη αμελητέας μάζας του \foreignlanguage{english}{b} κουάρκ και των διαφορετικών ενεργειακών κλιμάκων της ισχυρής αλληλεπίδρασης στις οποίες παράγονται τα \foreignlanguage{english}{top} και \foreignlanguage{english}{b} κουάρκς. Συνεπώς, η εύρεση κατάλληλων ενεργειακών κλιμάκων για τον υπολογισμό των στοιχείων πίνακα \foreignlanguage{english}{t$\overline{\text{t}}+$b$\overline{\text{b}}$} και η απρόσκοπτη σύνδεσή τους με τους πίδακες παρτονίων \foreignlanguage{english}{(PS)} και τις συναρτήσεις κατανομής παρτονίων \foreignlanguage{english}{(PDF)} γίνεται δύσκολη υπόθεση. Αβεβαιότητες που σχετίζονται με την επιλογή κλιμάκων επανακανονικοποίησης και παραγοντοποίησης σε υπολογισμούς στοιχείων πινάκων σε \foreignlanguage{english}{NLO} τάξη της Κβαντικής Χρωμοδυναμικής \foreignlanguage{english}{(QCD)} μπορούν να οδηγήσουν σε αβεβαιότητες έως και 50\% στις προβλέψεις των μετρήσεων περιεκτικών και διαφορικών ενεργών διατομών για την παραγωγή \foreignlanguage{english}{t$\overline{\text{t}}+$b$\overline{\text{b}}$}. Η βελτιωμένη γνώση αυτής της διεργασίας είναι, συνεπώς, πολύ σημαντική για τον έλεγχο των επιλογών κλιμάκων που γίνονται για τις πλέον σύγχρονες προσομοιώσεις. Ο κύριος στόχος αυτής της διπλωματικής εργασίας είναι η βελτιστοποίηση της προσομοίωσης της παραγωγής  \foreignlanguage{english}{t$\overline{\text{t}}+$b$\overline{\text{b}}$}, ενός καθοριστικού υποβάθρου σε πολλές αναζητήσεις και μετρήσεις. Ειδικότερα, η διεργασία  \foreignlanguage{english}{t$\overline{\text{t}}+$b$\overline{\text{b}}$} αποτελεί ένα ισχυρό υπόβαθρο για την συσχετισμένη παραγωγή ενός ζεύγους \foreignlanguage{english}{top} κουάρκς και ενός μποζονίου \foreignlanguage{english}{Higgs (t$\overline{\text{t}}$H)}, όπου το μποζόνιο \foreignlanguage{english}{Higgs} διασπάται σε ένα ζεύγος \foreignlanguage{english}{b} ή \foreignlanguage{english}{c} κουάρκς \foreignlanguage{english}{(H $\rightarrow$ b$\overline{\text{b}}/$c$\overline{\text{c}}$)}. Η καλύτερη κατανόηση της παραγωγής  \foreignlanguage{english}{t$\overline{\text{t}}+$b$\overline{\text{b}}$} θα συμβάλλει στην μείωση των αβεβαιοτήτων της διεργασίας \foreignlanguage{english}{t$\overline{\text{t}}$H(b$\overline{\text{b}}$/c$\overline{\text{c}}$)}, αυξάνοντας την ευαισθησία αναζήτησης της σύζευξης \foreignlanguage{english}{Yukawa} του \foreignlanguage{english}{c} κουάρκ, η οποία είναι μία σημαντική παράμετρος του Καθιερωμένου Προτύπου, όντας σύζευξη \foreignlanguage{english}{Yukawa} ενός κουάρκ δεύτερης γενιάς.}}
\selectlanguage{english}\fontencoding{OT1}\selectfont
